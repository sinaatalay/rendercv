% ======================================================================
% scrlttr2-experts-en.tex
% Copyright (c) Markus Kohm, 2002-2022
%
% This file is part of the LaTeX2e KOMA-Script bundle.
%
% This work may be distributed and/or modified under the conditions of
% the LaTeX Project Public License, version 1.3c of the license.
% The latest version of this license is in
%   http://www.latex-project.org/lppl.txt
% and version 1.3c or later is part of all distributions of LaTeX 
% version 2005/12/01 or later and of this work.
%
% This work has the LPPL maintenance status "author-maintained".
%
% The Current Maintainer and author of this work is Markus Kohm.
%
% This work consists of all files listed in MANIFEST.md.
% ======================================================================
%
% Chapter about scrlttr2 of the KOMA-Script guide expert part
% Maintained by Markus Kohm
%
% ============================================================================

\KOMAProvidesFile{scrlttr2-experts-en.tex}
                 [$Date: 2022-06-05 12:40:11 +0200 (So, 05. Jun 2022) $
                  KOMA-Script guide (chapter: scrlttr2 for experts)]

\translator{Harald Bongartz\and Georg Grandke\and Raimund Kohl\and Jens-Uwe
  Morawski\and Stephan Hennig\and Gernot Hassenpflug\and Markus Kohm\and
  Karl Hagen}

\chapter{Additional Information about the \Class{scrlttr2} Class and the
 \Package{scrletter} Package}
\labelbase{scrlttr2-experts}

\BeginIndexGroup
\BeginIndex{Class}{scrlttr2}
This chapter provides additional information about the \KOMAScript{} class
\Class{scrlttr2}. \iffree{Some parts of the chapter are found only in the
  German \KOMAScript{} book \cite{book:komascript}. This should not be a
  problem, because the}{The} average user, who only wants to use the class
or package, will not normally need this information. Part of this information
is addressed to users for whom the default options are insufficient. Thus,
for example, the first section describes in detail the pseudo-lengths that
specify the letterhead page and which can be used be used to modify the its
layout.%
\iffalse% Es wird Zeit das komplett rauszuwerfen!
  In addition, this chapter also describes features that exist only to
  provide compatibility with the deprecated \Class{scrlettr} class. It also
  explains in detail how to convert a document from this obsolete class
  to the current  class.
\fi

\BeginIndex{Package}{scrletter}%
Starting with \KOMAScript~3.15\ChangedAt[2014/11]{v3.15}{\Package{scrletter}},
you can use the \Package{scrletter} package with one of the \KOMAScript{}
classes \Class{scrartcl}, \Class{scrreprt}, or \Class{scrbook}. It provides
nearly all the features of \Class{scrlttr2} for those classes. There are,
however, a few differences described later in this chapter.%


\section{Variables for Experienced Users}
\seclabel{variables}
\BeginIndexGroup
\BeginIndex{}{variables}

\KOMAScript{} provides commands not only to use predefined variables but also
to define new variables or to change their automatic use within the reference
line\Index{reference line}.

\begin{Declaration}
  \Macro{newkomavar}\OParameter{description}\Parameter{name}%
  \Macro{newkomavar*}\OParameter{description}\Parameter{name}%
  \Macro{addtoreffields}\Parameter{name}
  \Macro{removereffields}%
  \Macro{defaultreffields}%
\end{Declaration}
\Macro{newkomavar} defines a new variable. This variable is referenced as
\PName{name}. Optionally, you can define a \PName{description} for the
\PName{name} variable. Unlike the \PName{name}, the \PName{description} is not
used to reference a variable. Instead, the \PName{description} acts as a
supplement to the content of a variable that can be printed as a label along
with its content.

You can use the \Macro{addtoreffields} command to add the \PName{name}
variable to the reference line\Index{reference line} (see
\autoref{sec:scrlttr2.firstpage}, \DescPageRef{scrlttr2.option.refline}). The
\PName{description} and the content of the variable are added to the end of
the reference line. The starred version \Macro{newkomavar*} is similar to the
unstarred version but also calls the \Macro{addtoreffields} command. Thus, the
starred version automatically adds the variable to the reference line.
\begin{Example}
  Suppose you need an additional field for a telephone extension in the
  reference line. You can define this field with
\begin{lstcode}
  \newkomavar[Extension]{myphone}
  \addtoreffields{myphone}
\end{lstcode}
  or more concisely with
\begin{lstcode}
  \newkomavar*[Extension]{myphone}
\end{lstcode}
\end{Example}
When\textnote{Attention!} you define a variable for the reference line, you
should always give it a description.

You can use the \Macro{removereffields} command to remove all variables from
the reference field. This includes the predefined variables of the class. The
reference line is then empty. This can be useful, for example, if you wish to
change the order of the variables in the reference fields line.

The \Macro{defaultreffields} command resets the reference fields line to its
predefined format. In doing so, all custom-defined variables are removed from
the reference fields line.

You\textnote{Attention!} should not add the date to the reference line with
the \Macro{addtoreffields} command. Instead you should use the
\DescRef{scrlttr2.option.refline}%
\important{\OptionValueRef{scrlttr2}{refline}{dateleft}\\
\OptionValueRef{scrlttr2}{refline}{dateright}\\
\OptionValueRef{scrlttr2}{refline}{nodate}}%
\IndexOption{refline~=\textKValue{dateleft}}%
\IndexOption{refline~=\textKValue{dateright}}%
\IndexOption{refline~=\textKValue{nodate}} option to select whether the date
should appear on the left or right side of the reference line, or not at all.
These settings also affect the position of the date when no reference line is
used.%
%
\EndIndexGroup


\begin{Declaration}
  \Macro{usekomavar}\OParameter{command}\Parameter{name}%
  \Macro{usekomavar*}\OParameter{command}\Parameter{name}
\end{Declaration}
The \DescRef{scrlttr2.cmd.usekomavar} and \DescRef{scrlttr2.cmd.usekomavar*}
commands are, like all commands where a starred version exists or which can
take an optional argument, not fully expandable. Nevertheless, if you use them
within \DescRef{scrlttr2.cmd.markboth}\IndexCmd{markboth},
\DescRef{scrlttr2.cmd.markright}\IndexCmd{markright} or similar commands, you
need not insert \Macro{protect}\IndexCmd{protect} beforehand. Of course
this is also true for
\DescRef{scrlayer-scrpage.cmd.markleft}\IndexCmd{markleft} if you use the
\hyperref[cha:scrlayer-scrpage]{\Package{scrlayer-scrpage}}%
\IndexPackage{scrlayer-scrpage} package. These\textnote{Attention!} commands
cannot be used within commands that directly affect their argument, such as
\Macro{MakeUppercase}\important{\Macro{MakeUppercase}}%
\IndexCmd{MakeUppercase}. To avoid this problem you can use commands like
\Macro{MakeUppercase} as an optional argument to \Macro{usekomavar} or
\Macro{usekomavar*}. Then you will get the upper-case content of a variable
with
\begin{lstcode}[escapeinside=`']
  \usekomavar[\MakeUppercase]{`\PName{Name}'}
\end{lstcode}
%
\EndIndexGroup


\begin{Declaration}
  \Macro{Ifkomavarempty}\Parameter{name}\Parameter{true}\Parameter{false}%
  \Macro{Ifkomavarempty*}\Parameter{name}\Parameter{true}\Parameter{false}
\end{Declaration}
It is important to know that the content of the variable will be expanded as
far as this is possible with \Macro{edef}. If this results in spaces or
unexpandable macros like \Macro{relax}, the result will be not empty even
where the use of the variable would not result in any visible output.

Once\textnote{Attention!} again, this command cannot be used as the argument
of \Macro{MakeUppercase}\IndexCmd{MakeUppercase} or similar commands However,
it is robust enough to be used as the argument of
\DescRef{scrlttr2.cmd.markboth} or \DescRef{scrlttr2.cmd.footnote}, for
example.%
%
\EndIndexGroup


\begin{Declaration}
  \Macro{foreachkomavar}\Parameter{list of variables}\Parameter{command}
  \Macro{foreachnonemptykomavar}\Parameter{list of variables}\Parameter{command}
  \Macro{foreachemptykomavar}\Parameter{list of variables}\Parameter{command}
  \Macro{foreachkomavarifempty}\Parameter{list of variables}
                               \Parameter{then-code}\Parameter{else-code}
\end{Declaration}
The\ChangedAt{v3.27}{\Class{scrlttr2}\and \Package{scrletter}} 
\Macro{foreachkomavar} command executes the specified \PName{command} for each
variable in the comma-separated \PName{list of variables}. The name of each
variable is added as parameter to the \PName{command}.%

The \Macro{foreachnonemptykomavar} command does the same but only for those
variables that are not empty in sense of
\DescRef{\LabelBase.cmd.Ifkomavarempty}. Empty variables in the \PName{list of
  variables} are ignored.

By contrast, the \Macro{foreachemptykomavar} command executes the
\PName{command} only for variables that are empty in sense of
\DescRef{\LabelBase.cmd.Ifkomavarempty}. Accordingly, non-empty variables are
ignored.

The \Macro{foreachkomavarifempty} command is a kind of combination of the two
previously described commands. It executes the \PName{then-command} only for
those variables in the \PName{list of variables} that are empty, and the
\PName{else-command} for the non empty variables. As with \PName{command}, the
name of each variable is added as a parameter in both cases.%
\EndIndexGroup
%
\EndIndexGroup


\section{Additional Information about Page Styles}
\seclabel{pagestyleatscrletter}
\BeginIndexGroup
\BeginIndex{}{page>style}

\LoadNonFree{scrlttr2-experts}{0}%
\EndIndexGroup

\iffalse% Wurde bereits in Kapitel 4.21 behandelt
\section{Differences in How \Package{scrletter} Handles \File{lco} Files}
\seclabel{lcoatscrletter}
\BeginIndexGroup
\BeginIndex{File}{lco}
\BeginIndex{}{lco file=\File{lco} file}

As\ChangedAt{v3.15}{\Package{scrletter}} explained in
\autoref{sec:scrlttr2.lcoFile}, \Class{scrlttr2} can load \File{lco} files via
the optional argument of \Macro{documentclass}. The \Package{scrletter} package
does not support this.

\begin{Declaration}
  \Macro{LoadLetterOption}\Parameter{name}%
  \Macro{LoadLetterOptions}\Parameter{list of names}
\end{Declaration}
For \Class{scrlttr2}, load \File{lco} files with
\DescRef{scrlttr2.cmd.LoadLetterOption} or
\DescRef{scrlttr2.cmd.LoadLetterOptions} is only a recommendation. For
\Package{scrletter}, it is mandatory. Of course, you can only load the
\File{lco} files after you load \Package{scrletter}.
%
\EndIndexGroup
%
\EndIndexGroup
\fi


\section{\File{lco} Files for Experienced Users}
\seclabel{lcoFile}
\BeginIndexGroup
\BeginIndex{File}{lco}
\BeginIndex{}{lco file=\File{lco} file}

\BeginIndexGroup%
\BeginIndex{}{paper>format}%
Although you can use any paper size that the
\hyperref[cha:typearea]{\Package{typearea}}%
\IndexPackage{typearea} package can configure, the output of the letterhead
page may produce undesirable results with some formats. Unfortunately, there
are no general rules to calculate the position of the address fields and the
like for every available paper size. Instead, different parameter sets are
needed for different paper sizes.%

%\subsection{Verifying the Paper Size}
%\seclabel{papersize}

At present parameter sets and \File{lco} files exist only for A4-sized and
letter-sized paper. Theoretically, however, the \Class{scrlttr2} class can
support many more paper sizes. Therefore, it's necessary to verify that the
correct paper size is used. This is even more true if you use
\Package{scrletter}, since the paper size depends on the class you use.

\begin{Declaration}
  \Macro{LetterOptionNeedsPapersize}%
    \Parameter{option name}\Parameter{paper size}
\end{Declaration}
To provide at least a warning when another \PName{paper size} is used, you can
find a \Macro{LetterOptionNeedsPapersize} command in every \File{lco} file
distributed with {\KOMAScript}. The first argument is the name of the
\File{lco} file without the ``\File{.lco}'' suffix. The second argument is the
paper size for which the \File{lco} file is designed.

If several \File{lco} files are loaded in succession, a
\Macro{LetterOptionNeedsPapersize} command can be contained in each of them,
but the \DescRef{scrlttr2.cmd.opening} command will only check the last given
\PName{paper size}. As the following example shows, an experienced user can
thus easily write \File{lco} files with parameter sets for other paper sizes.
\iffalse% Umbruchoptimierung
If you do not plan to set up such \File{lco} files yourself, you can just
forget about this option and skip the example.%
\fi
\begin{Example}
  Suppose you use A5-sized paper in normal, that is upright or portrait,
  orientation for your letters. Let's assume that you want to put them into
  standard C6 window envelopes. In that case, the position of the address
  field would be the same as for a standard letter on A4-sized paper. The main
  difference is that A5 paper needs only one fold. So you want to disable the
  top and bottom fold marks. You can do this, for example, by placing the
  marks outside the paper area.
\begin{lstcode}
  \ProvidesFile{a5.lco}
               [2002/05/02 letter class option]
  \LetterOptionNeedsPapersize{a5}{a5}
  \setplength{tfoldmarkvpos}{\paperheight}
  \setplength{bfoldmarkvpos}{\paperheight}
\end{lstcode}
  Of course, it would be more elegant to deactivate the marks with the
  \DescRef{scrlttr2.option.foldmarks} option. In addition, you must adjust the
  position of the footer, that is, the \PLength{firstfootvpos} pseudo-length.
  I leave it to the reader to find an appropriate value. When using such an
  \File{lco} file, you must declare other \File{lco} file options like
  \File{SN} before you load ``\File{a5.lco}''.
\end{Example}
%
\EndIndexGroup%
\EndIndexGroup%
\vskip-\ht\strutbox% Wegen Beispiel am Ende der Erklaerung


%\subsection{Visualizing Positions}
%\seclabel{visualize}
%\BeginIndexGroup
\begin{Declaration}
  \File{visualize.lco}
\end{Declaration}
\BeginIndex{Option}{visualize}%
If you develop your own \File{lco} file, for example to modify the positions
of various fields on the letterhead page because your own desires or
requirements, it is helpful if you can make at least some elements directly
visible. The \File{lco} file
\File{visualize.lco}\ChangedAt{v3.04}{\Class{scrlttr2}} exists for this
purpose. You can load this file as you would any other \File{lco} file. But
this \emph{letter class options} file must be loaded in the document preamble,
and its effects cannot be deactivated. The \File{lco} file uses the
\Package{eso-pic}\IndexPackage{eso-pic}%
\important[i]{\Package{eso-pic}\\\Package{graphicx}} and
\Package{graphicx}\IndexPackage{graphicx} packages, which are not part of
\KOMAScript.


\begin{Declaration}
  \Macro{showfields}\Parameter{field list}
\end{Declaration}
This command makes the space occupied by the fields on the letterhead page
visible. The \PName{field list} argument is a comma-separated list of fields
to be shown. The following fields are supported:
\begin{labeling}[~--]{\PValue{location}}
\item[\PValue{test}] is a 10\Unit{cm} by 15\Unit{cm} test field, 1\Unit{cm}
  from the top and left edges of the paper. This field exists for debugging.
  You can use it as a benchmark to check whether the measurements have been
  distorted during the creation of the document.
\item[\PValue{head}] is the header area of the letterhead page. This field
  is open at the bottom.
\item[\PValue{foot}] is the footer area of the letterhead page. This field is
  open at the top.
\item[\PValue{address}] is the address window area used by window envelopes.
\item[\PValue{location}] is the field for the extra sender information.
\item[\PValue{refline}] is the reference line. This field is open at the
  bottom.
\end{labeling}%
\BeginIndex{FontElement}{field}\LabelFontElement{field}%
You can change the colour of the visualisation with
the\DescRef{scrlttr2.cmd.setkomafont} and \DescRef{scrlttr2.cmd.addtokomafont}
(see \autoref{sec:scrlttr2.textmarkup},
\DescPageRef{scrlttr2.cmd.setkomafont}) commands using the
\FontElement{field}\important{\FontElement{field}} element. The default is
\Macro{normalcolor}.%
\EndIndex{FontElement}{field}%
%
\EndIndexGroup


\begin{Declaration}
  \Macro{setshowstyle}\Parameter{style}%
  \Macro{edgesize}
\end{Declaration}
By default, \File{visualize.lco} indicates the individual areas with
frames\important{\PValue{frame}}, which corresponds to the \PName{style}
\PValue{frame}. Areas open at top or bottom are not completely framed but have
an open edge with with small arrows pointing up or down.
Alternatively\important{\PValue{rule}}, you can use the \PName{style}
\PValue{rule}. In this case, the area is highlighted by a background colour.
It isnot possible to distinguish open and closed areas. Instead a minimal
height will be used for open areas. The third\important{\PValue{edges}}
available \PName{style} is \PValue{edges}, which shows the corners of the
areas. The corner marks at the open edge of open areas will be omitted. The
size of two edges of the corner marks are given by the \Macro{edgesize} macro
with a default of 1\Unit{ex}.%
\EndIndexGroup


\begin{Declaration}
  \Macro{showenvelope}\AParameter{width\textup{,}height}%
                      \AParameter{h-offset\textup{,}v-offset}%
                      \OParameter{instructions}%
  \Macro{showISOenvelope}\Parameter{format}\OParameter{instructions}%
  \Macro{showUScommercial}\Parameter{format}\OParameter{instructions}%
  \Macro{showUScheck}\OParameter{instructions}%
  \Macro{unitfactor}
\end{Declaration}
If you have loaded \File{visualize.lco}, you can use these commands to output
a page with a drawing of an envelope. The envelope drawing is always rotated
by 90\textdegree{} on a separate page and printed in 1:1~scale. The addressee
window is generated automatically from the current data for the address
position of the letterhead page: \PLength{toaddrvpos}, \PLength{toaddrheight},
\PLength{toaddrwidth}, and \PLength{toaddrhpos}. To do so requires knowing how
much smaller the folded letter pages are than the width and height of the
envelope. If you do not specify these two values, \PName{h-offset} and
\PName{v-offset}, when calling \Macro{showenvelope}, they are calculated from
the fold marks and the paper size itself.

The \Macro{showISOenvelope}, \Macro{showUScommercial}, and \Macro{showUScheck}
commands are based on \Macro{showenvelope}. With \Macro{showISOenvelope}, you
can create ISO-envelopes in C4, C5, C5/6, DL (also known as C5/6) or C6
\PName{format}. With \Macro{showUScommercial}, you can create a US commercial
envelope in the 9 or 10 \PName{format}. You can use \Macro{showUScheck} for
envelopes in US check format.

\BeginIndex{FontElement}{letter}\LabelFontElement{letter}%
The position of the letterhead page inside the envelope is indicated with
dashed lines. You can change the colour of these lines with the
\DescRef{scrlttr2.cmd.setkomafont} and \DescRef{scrlttr2.cmd.addtokomafont}
(see \autoref{sec:scrlttr2.textmarkup},
\DescPageRef{scrlttr2.cmd.setkomafont}) using the
\FontElement{letter}\important{\FontElement{letter}} element. The default is
\Macro{normalcolor}.%
\EndIndex{FontElement}{letter}%

\BeginIndex{FontElement}{measure}\LabelFontElement{measure}%
The envelope drawing will be provided with dimensions automatically. You can
change the colour of these dimension labels with the commands
\DescRef{scrlttr2.cmd.setkomafont} and \DescRef{scrlttr2.cmd.addtokomafont}
(see \autoref{sec:scrlttr2.textmarkup},
\DescPageRef{scrlttr2.cmd.setkomafont}) using the
\FontElement{measure}\important{\FontElement{measure}} element. The default is
\Macro{normalcolor}. The dimensions are given in multiples of
\Length{unitlength}, with an accuracy of $1/\Macro{unitfactor}$, where the
accuracy of \TeX{} arithmetic is the actual limits. The default is 1. You can
redefine \Macro{unitfactor} using \Macro{renewcommand}.%
\EndIndex{FontElement}{measure}%

\begin{Example}
  You are generating a sample letter using the ISO-A4 format. The supported
  fields should be marked with yellow borders to check their position.
  Furthermore, the position of the window for a DL-size envelope should be
  checked with drawing. The dimension lines in this drawing should be red, and
  the numbers should use a smaller font, with the dimensions printed in cm
  with an accuracy of 1\Unit{mm}. The dashed letterhead page in the envelope
  should be coloured green.
\begin{lstcode}
  \documentclass[visualize]{scrlttr2}
  \usepackage{xcolor}
  \setkomafont{field}{\color{yellow}}
  \setkomafont{measure}{\color{red}\small}
  \setkomafont{letter}{\color{green}}
  \showfields{head,address,location,refline,foot}
  \usepackage[british]{babel}
  \usepackage{lipsum}
  \begin{document}
  \setkomavar{fromname}{Joe Public}
  \setkomavar{fromaddress}{2 Valley\\
                           SAMPLEBY\\
                           ZY32 1XW}
  \begin{letter}{%
      1 Hillside\\
      SAMPLESTEAD\\
      WX12 3YZ%
    }
  \opening{Hello,}
  \lipsum[1]
  \closing{Good bye}
  \end{letter}
  \setlength{\unitlength}{1cm}
  \renewcommand*{\unitfactor}{10}
  \showISOenvelope{DL}
  \end{document}
\end{lstcode}
  This will show the letterhead page as the first page and the drawing of the
  envelope on the second page.
\end{Example}

Note that poorly chosen combinations of \Length{unitlength} and
\Macro{unitfactor} can quickly lead to a \TeX{} \emph{arithmetic overflow}
error. The dimension numbers shown may also differ slightly from the actual
values. Neither are errors in \Option{visualize} but merely implementation
limitations of \TeX.
%
\EndIndexGroup
%
\EndIndexGroup
%
\EndIndexGroup


\section{Language Support}
\seclabel{languages}%
\BeginIndexGroup
\BeginIndex{}{languages}%

% TODO: New translation of the shorter into in German manual should be used.
% :ODOT
The \Class{scrlttr2} class and the \Package{scrletter} package support many
languages. These include German\Index{language>German} (\PValue{german} for
the old German orthography, \PValue{ngerman} for the new orthography;
\PValue{austrian} for Austrian with the old German orthography,
\PValue{naustrian}\ChangedAt{v3.09}{\Class{scrlttr2}} for Austrian with the
new orthography; and \PValue{nswissgerman}\ChangedAt{v3.13}{\Class{scrlttr2}}
for Swiss German with the new orthogrphy, \PValue{swissgerman} for Swiss
German with the old orthography), English\Index{language>English} (among
others, \PValue{english} without specification as to whether American or
British should be used, \PValue{american} and \PValue{USenglish} for American
English, and \PValue{british} and \PValue{UKenglish} for British English),
French\Index{language>French}, Italian\Index{language>Italian},
Spanish\Index{language>Spanish}, Dutch\Index{language>Dutch},
Croatian\Index{language>Croatian}, Finnish\Index{language>Finnish},
Norwegian\Index{language>Norwegian}\ChangedAt{v3.02}{\Class{scrlttr2}},
Swedish\Index{language>Swedish}\ChangedAt{v3.08}{\Class{scrlttr2}},
Polish\ChangedAt{v3.13}{\Class{scrlttr2}},
Czech\ChangedAt{v3.13}{\Class{scrlttr2}}, and Slovak.

You can switch languages using the \Package{babel}\IndexPackage{babel} package
(see \cite{package:babel}) with the \Macro{selectlanguage}\Parameter{language}
command.  Other packages like \Package{german}\IndexPackage{german} (see
\cite{package:german}) and \Package{ngerman}\IndexPackage{ngerman} (see
\cite{package:ngerman}) also define this command.  As a rule though, the
language selection occurs immediately as a direct consequence of loading such
a package.
\iffalse% Umbruchkorrekturtext
For details, please refer to the documentation of the relevant packages.
\fi

There\textnote{Attention!} is one more point to note about language-switching
packages. The
\Package{french}\IndexPackage{french}\important{\Package{french}} package (see
\cite{package:french}) makes changes well beyond redefining the terms in
\autoref{tab:\LabelBase.languageTerms}. For instance, it redefines the
\DescRef{scrlttr2.cmd.opening} command, since the package simply assumes that
\DescRef{scrlttr2.cmd.opening} is always defined as it is in the standard
\Class{letter} class. This, however, is not the case with \KOMAScript{}. The
\Package{french} package thus overwrites the definition and does not work
correctly with \KOMAScript. I regard this as a fault in the \Package{french}
package which, although reported decades ago, was unfortunately never
eliminated.

If you use the \Package{babel}\IndexPackage{babel} package to switch to
\PValue{french}, problems can occasionally occur. With \Package{babel},
however, you can usually deactivate changes to a language in a targeted
manner.%
\iffalse% This is actually outdated!
\ If the package \Package{french} is not installed, the problem with
  \Package{babel} does not arise. Similarly, the problem usually does not
  exist when you use \Package{babel} with other varieties of French such as
  \PValue{acadian}, \PValue{canadien}, \PValue{francais} or \PValue{frenchb}
  instead of \PValue{french}.
\fi

\iffalse% This is also outdated!
  However, with \Package{babel} version 3.7j and above, this problem only occurs
  if the option explicitly indicates that \Package{babel} should use the
  \Package{french} package.
%
\iftrue
  If you cannot be sure you are not using an old version of \Package{babel}, I
  recommend you use
\begin{lstcode}
  \usepackage[...,frenchb,...]{babel}
\end{lstcode}
  to select French.
    \iffalse %
      If necessary, you can still switch to French with
      \Macro{selectlanguage}\PParameter{french}.%
    \fi%
\fi 
\fi  

\iffalse
  It cannot be ruled out that similar problems will not occur with other
  languages or packages. For German and English, however, there are currently
  no known problems with the \Package{babel} package.
\fi


\begin{Declaration}
  \Macro{yourrefname}%
  \Macro{yourmailname}%
  \Macro{myrefname}%
  \Macro{customername}%
  \Macro{invoicename}%
  \Macro{subjectname}%
  \Macro{ccname}%
  \Macro{enclname}%
  \Macro{headtoname}%
  \Macro{headfromname}%
  \Macro{datename}%
  \Macro{pagename}%
  \Macro{mobilephonename}%
  \Macro{phonename}%
  \Macro{faxname}%
  \Macro{emailname}%
  \Macro{wwwname}%
  \Macro{bankname}
\end{Declaration}
These commands contain the language-dependent terms. These\important[i]{%
  \DescRef{scrbase.cmd.newcaptionname}\\
  \DescRef{scrbase.cmd.renewcaptionname}\\
  \DescRef{scrbase.cmd.providecaptionname}} definitions can be
modified to support a new language or for your private customization, as
described in
\autoref{sec:scrbase.languageSupport}. \KOMAScript{} sets these terms only in
\Macro{begin}\PParameter{document}. Therefore they are not available in the
preamble and cannot be redefined there. The default settings for
\Option{english} and \Option{ngerman} are listed in
\autoref{tab:\LabelBase.languageTerms}.%
\EndIndexGroup
%\iffree{}{\clearpage}% Siehe Kommentar zur Tabelle

\begin{Declaration}
  \Macro{captionsacadian}
  \Macro{captionsamerican}
  \Macro{captionsaustralien}
  \Macro{captionsaustrian}
  \Macro{captionsbritish}
  \Macro{captionscanadian}
  \Macro{captionscanadien}
  \Macro{captionscroatian}
  \Macro{captionsczech}
  \Macro{captionsdutch}
  \Macro{captionsenglish}
  \Macro{captionsfinnish}
  \Macro{captionsfrancais}
  \Macro{captionsfrench}
  \Macro{captionsgerman}
  \Macro{captionsitalian}
  \Macro{captionsnaustrian}
  \Macro{captionsnewzealand}
  \Macro{captionsngerman}
  \Macro{captionsnorsk}
  \Macro{captionsnswissgerman}
  \Macro{captionspolish}
  \Macro{captionsslovak}
  \Macro{captionsspanish}
  \Macro{captionsswedish}
  \Macro{captionsswissgerman}
  \Macro{captionsUKenglish}
  \Macro{captionsUSenglish}
\end{Declaration}
If you change the language of a letter, the language-dependent terms listed in
\autoref{tab:\LabelBase.languageTerms},
\autopageref{tab:\LabelBase.languageTerms} are redefined using these
commands. If your language-switching package does not support this, you can
also use the above commands directly.
%
\begin{table}
  \begin{minipage}{\textwidth}
  \centering
%    \KOMAoptions{captions=topbeside}%
%    \setcapindent{0pt}%
   \caption[{%
        Defaults for language-dependent terms
      }]{%
        Defaults for language-dependent terms for the languages
        \Option{english} and \Option{ngerman}, if they are not already defined
        by the packages used for language switching%
        \label{tab:\LabelBase.languageTerms}%
      }[l]
      \begin{tabular}[t]{lll}
        \toprule
        Command         & \Option{english} & \Option{ngerman} \\
        \midrule
        \Macro{bankname}     & Bank account   & Bankverbindung \\
        \Macro{ccname}\footnotemark[1]       & cc             & Kopien an \\
        \Macro{customername} & Customer no.   & Kundennummer \\
        \Macro{datename}     & Date           & Datum \\
        \Macro{emailname}    & Email          & E-Mail \\
        \Macro{enclname}\footnotemark[1]     & encl           & Anlagen \\
        \Macro{faxname}      & Fax            & Fax \\
        \Macro{headfromname} & From           & Von \\
        \Macro{headtoname}\footnotemark[1]   & To             & An \\
        \Macro{invoicename}  & Invoice no.    & Rechnungsnummer \\
        \Macro{myrefname}    & Our ref.       & Unser Zeichen \\
        \Macro{pagename}\footnotemark[1]     & Page           & Seite \\
        \Macro{mobilephonename} & Mobile phone & Mobiltelefon \\
        \Macro{phonename}    & Phone          & Telefon \\
        \Macro{subjectname}  & Subject        & Betrifft \\
        \Macro{wwwname}      & Url            & URL \\
        \Macro{yourmailname} & Your letter of & Ihr Schreiben vom\\
        \Macro{yourrefname}  & Your ref.      & Ihr Zeichen \\
        \bottomrule
      \end{tabular}
      \deffootnote{1em}{1em}{1\ }% brutal aber effektiv
      \footnotetext[1000]{Normally these terms are defined by language
        packages like \Package{babel}. In this case, \KOMAScript{} does not
        redefine them. The actual wording may therefore differ and can
        be found in the documentation for the language package used.}
%    \end{captionbeside}
  \end{minipage}
\end{table}
%
\EndIndexGroup
\iffree{}{\clearpage}% Umbruchkorrektur

\begin{Declaration}
  \Macro{dateacadian}
  \Macro{dateamerican}
  \Macro{dateaustralien}
  \Macro{dateaustrian}
  \Macro{datebritish}
  \Macro{datecanadian}
  \Macro{datecanadien}
  \Macro{datecroatian}
  \Macro{dateczech}
  \Macro{datedutch}
  \Macro{dateenglish}
  \Macro{datefinnish}
  \Macro{datefrancais}
  \Macro{datefrench}
  \Macro{dategerman}
  \Macro{dateitalian}
  \Macro{datenaustrian}
  \Macro{datenewzealand}
  \Macro{datengerman}
  \Macro{datenorsk}
  \Macro{datenswissgerman}
  \Macro{datepolish}
  \Macro{dateslovak}
  \Macro{datespanish}
  \Macro{dateswedish}
  \Macro{dateswissgerman}  
  \Macro{dateUKenglish}
  \Macro{dateUSenglish}
\end{Declaration}
Depending on the language used, the numerical representation of the date\Index{date} (see option
\DescRef{scrlttr2.option.numericaldate} in \autoref{sec:scrlttr2.firstpage},
\DescPageRef{scrlttr2.option.numericaldate}) is formatted in various ways.
The exact format can be found in \autoref{tab:date}.%
\EndIndexGroup
%
\EndIndexGroup
\iffree{}{\clearpage}% Umbruchkorrektur zwecks Ausgabe der Tabellen

\begin{table}[!tp]% Umbruchoptimierung
%  \centering
  \KOMAoptions{captions=topbeside}%
  \setcapindent{0pt}%
%  \caption
  \begin{captionbeside}{Language-dependent forms of the date}[l]
  \begin{tabular}[t]{ll}
    \toprule
    Command & Date Example \\
    \midrule
      \Macro{dateacadian}   & 24.\,12.\,1993\\
      \Macro{dateamerican}  & 12/24/1993\\
      \Macro{dateaustralien}& 24/12/1993\\
      \Macro{dateaustrian}  & 24.\,12.\,1993\\
      \Macro{datebritish}   & 24/12/1993\\
      \Macro{datecanadian}  & 1993/12/24\\
      \Macro{datecanadien}  & 1993/12/24\\
      \Macro{datecroatian}  & 24.\,12.\,1993.\\
      \Macro{dateczech}     & 24.\,12.\,1993\\
      \Macro{datedutch}     & 24.\,12.\,1993\\
      \Macro{dateenglish}   & 24/12/1993\\
      \Macro{datefinnish }  & 24.12.1993.\\
      \Macro{datefrancais}  & 24.\,12.\,1993\\
      \Macro{datefrench}    & 24.\,12.\,1993\\
      \Macro{dategerman}    & 24.\,12.\,1993\\
      \Macro{dateitalian}   & 24.\,12.\,1993\\
      \Macro{datenaustrian} & 24.\,12.\,1993\\
      \Macro{datenewzealand}& 24/12/1993\\
      \Macro{datengerman}   & 24.\,12.\,1993\\
      \Macro{datenorsk}     & 24.12.1993\\
      \Macro{datenswissgerman}   & 24.\,12.\,1993\\
      \Macro{datepolish}    & 24.\,12.\,1993\\
      \Macro{dateslovak}    & 24.\,12.\,1993\\
      \Macro{datespanish}   & 24.\,12.\,1993\\
      \Macro{dateswedish}   & 24/12 1993\\
      \Macro{dateswissgerman}    & 24.\,12.\,1993\\
      \Macro{dateUKenglish} & 24/12/1993\\
      \Macro{dateUSenglish} & 12/24/1993\\
    \bottomrule
  \end{tabular}
  \end{captionbeside}
  \label{tab:date}
\end{table}
%
\section{Obsolete Commands}
\seclabel{obsolete}
\BeginIndexGroup

\LoadNonFree{scrlttr2-experts}{1}
\EndIndexGroup
%
\iffalse% Es wird Zeit das komplett rauszuwerfen!
\section{From Obsolete \Class{scrlettr} to Current \Class{scrlttr2}}
\seclabel{fromscrlettr}

With\textnote{Attention!} the 2002 release of \Class{scrlttr2} (see
\autoref{cha:scrlttr2}), the old letter class, \Class{scrlettr}, became
obsolete, and it is no longer part of \KOMAScript. If you still need the class
or information about it, you will find it in
\cite{package:koma-script-obsolete}.

To facilitate the transition to the new class, there is the compatibility
option \Option{KOMAold}. Basically, all the older functionality still
exists in the new class.  Without \Option{KOMAold}, however, the user
interface and the defaults are different. More details on
this option are provided in \autoref{sec:scrlttr2.lcoFile},
\autoref{tab:lcoFiles}.

\LoadNonFree{scrlttr2-experts}{2}
%
\EndIndexGroup
\fi
%
\EndIndexGroup

\endinput

%%% Local Variables: 
%%% mode: latex
%%% TeX-master: "scrguide-en.tex"
%%% coding: utf-8
%%% ispell-local-dictionary: "en_GB"
%%% eval: (flyspell-mode 1)
%%% End: 
